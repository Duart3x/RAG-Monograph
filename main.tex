\documentclass[conference]{IEEEtran}
\IEEEoverridecommandlockouts
% The preceding line is only needed to identify funding in the first footnote. If that is unneeded, please comment it out.
\usepackage{cite}
\usepackage{amsmath,amssymb,amsfonts}
\usepackage{graphicx}
\usepackage{textcomp}
\usepackage{xcolor}
\usepackage{amsmath}
\usepackage{algorithm}
\usepackage{algorithmicx}
\usepackage{algpseudocode}
\usepackage{multirow}

\makeatletter
\def\BState{\State\hskip-\ALG@thistlm}
\makeatother

\def\BibTeX{{\rm B\kern-.05em{\sc i\kern-.025em b}\kern-.08em
    T\kern-.1667em\lower.7ex\hbox{E}\kern-.125emX}}
\begin{document}

\title{Retrieval Augmented Generation}

\author{\IEEEauthorblockN{Duarte Santos}
    \IEEEauthorblockA{\textit{DETI} \\
        \textit{Universidade de AVeiro}\\
        124376 - duartevsantos@ua.pt}}
% Monograph about RAG (Retrieval Augmented Generation)

\maketitle

\begin{abstract}
    Retrieval Augmented Generation (RAG) is a model that combines the strengths of retrieval-based and generation-based models (GPT \cite{gpt}, Llama \cite{llama}).
    It uses a retriever to find relevant information from a large corpus and a generator to produce the final output.
    This model has been shown to outperform previous models in several tasks, such as question answering and text summarization.
    In this monograph, I present an overview of the RAG model, its architecture, and its training process.
    It's also discussed the advantages and limitations of the model and present some of the most recent research in the field.
    Finally, I present some ideas for future research and discuss the potential impact of RAG on the field of natural language processing.
\end{abstract}

\begin{IEEEkeywords}
    Retrieval Augmented Generation, RAG, Natural Language Processing, Question Answering, Text Summarization
\end{IEEEkeywords}

\section{Introduction}


\section{Conclusion}

\bibliographystyle{IEEEtran}
\bibliography{refs}

\end{document}
